\documentclass[12pt]{article}
\usepackage[margin=1in]{geometry}          
\usepackage{graphicx}
\usepackage{comment}
\usepackage{amsthm, amsmath, amssymb}
\usepackage{mathtools}
\usepackage{setspace}\onehalfspacing
\usepackage[loose,nice]{units} %replace "nice" by "ugly" for units in upright fractions
 
\title{Statistics Basics Week 3 SGA}
\author{Yumen Cao}
\date{June 27th 2022}
 
\begin{document}
\maketitle

Let $X$ be the random variable that denotes the number of people who became happier after avoiding using social network for a week.

Let $H_0$ be probability of becoming happier is $=\frac{1}{2}$

Let $H_1$ be probability of becoming happier is  $>\frac{1}{2}$

Since there is no clear information of the distribution of $X$, we are going to assume that 
\[
X \sim Binomial (n, p)
\]
where $n=20$ and $p=\frac{1}{2}$

Then we can find the p-value:
\[
p(X\geq X_{obs}|H_0) = \\
p(X=20|H_0) + p(X=19|H_0) + p(X=18|H_0) + p(X=17|H_0) + p(X=16|H_0)
\]

\begin{align}
\begin{split}
p(X\geq X_{obs}|H_0) = &\\
&\binom{20}{20}(\frac{1}{2})^{20} (\frac{1}{2})^{0} + \\
&\binom{20}{19}(\frac{1}{2})^{19} (\frac{1}{2})^{1} + \\
&\binom{20}{18}(\frac{1}{2})^{18} (\frac{1}{2})^{2} + \\
&\binom{20}{17}(\frac{1}{2})^{17} (\frac{1}{2})^{3} + \\
&\binom{20}{16}(\frac{1}{2})^{16} (\frac{1}{2})^{4} \\
= & 0.0059
\end{split}
\end{align}

This means p-value < 0.05 and we can conclude that people are happier without social networks.

\end{document}