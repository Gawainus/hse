\documentclass[12pt]{article}
\usepackage[margin=1in]{geometry}          
\usepackage{graphicx}
\usepackage{comment}
\usepackage{amsthm, amsmath, amssymb}
\usepackage{mathtools}
\usepackage{setspace}\onehalfspacing
\usepackage[loose,nice]{units} %replace "nice" by "ugly" for units in upright fractions
 
\title{Probability Theory Graded Assignment Week 2 Problem 1}
\author{Yumen Cao}
\date{April 2022}
 
\begin{document}
\maketitle

Let $X$ be the random variable that denotes launches without insurance.

If the launch succeeds, $X=100$ and $P(X)=0.9$, else, $X=-200$ and $P(X)=0.1$

\begin{align}
E(X) &= 0.9*100 + 0.1* (-200) \\
&= 70
\end{align}

\begin{align}
E(X^2) &= 0.9*100^2 + 0.1* (-200)^2 \\
&= 13000
\end{align}


\begin{align}
Var(X) &= E(X^2) - E^2(X) \\
&= 13000 - 70^2 \\
&= 8100
\end{align}

Let $Y$ be the random variable that denotes launches with insurance. Regardless the launch result, we pay for the insurance.

If the launch succeeds, $=100 - 30 = 70$ and $P(Y)=0.9$, else, $=-200 + 200 -30 = -30$ and $P()=0.1$

\begin{align}
E(Y) &= 0.9*70 + 0.1* (-30) \\
&= 60
\end{align}

\begin{align}
E(Y^2) &= 0.9*70^2 + 0.1* (-30)^2 \\
&= 4500
\end{align}


\begin{align}
Var(Y) &= E(Y^2) - E^2(Y) \\
&= 4500 -60^2 \\
&= 900
\end{align}


To summarise, we have
\begin{center}
\begin{tabular}{||c| c |c||} 
 \hline
 Stats & $X$ & $Y$ \\ [0.5ex] 
 \hline
 \hline
 $E$ & 70 & 60 \\
 \hline
 $Var$ & 8100 & 900 \\
 \hline
\end{tabular}
\end{center}

 
 This means that buying insurance lowers the expectation of profit and {\bf does not make sense in the long run}. But if the company just started, it would be devastating to suffer successive losses without insurance. The low variance provided by the insurance could make the company cashflow {\bf more stable}.

\end{document}