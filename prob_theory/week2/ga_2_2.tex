\documentclass[12pt]{article}
\usepackage[margin=1in]{geometry}          
\usepackage{graphicx}
\usepackage{comment}
\usepackage{amsthm, amsmath, amssymb}
\usepackage{mathtools}
\usepackage{setspace}\onehalfspacing
\usepackage[loose,nice]{units} %replace "nice" by "ugly" for units in upright fractions
 
\title{Probability Theory Graded Assignment Week 2 Problem 1}
\author{Yumen Cao}
\date{April 2022}
 
\begin{document}
\maketitle

First let's find the PMF of the random variable $X$
\begin{center}
\begin{tabular}{||c| c |c||} 
 \hline
 Case & $X$ & $P(X)$ \\ [0.5ex] 
 \hline
 \hline
 Head and win & $100-10=90$ & $0.5*0.1 = 0.05$ \\
 \hline
  \hline
 Head and lose & $-10$ & $0.5*0.9 = 0.45$\\
 \hline
 Tail and win \$50 &  $50-20=30$ & $0.5*0.1 = 0.05$ \\
 \hline
  \hline
 Tail and win \$500 &  $500-20=480$ & $0.5*0.01 = 0.005$ \\
 \hline
  \hline
 Tail and lose &  $-20$ & $0.5*0.89 = 0.445$ \\
 \hline

\end{tabular}
\end{center}


 \begin{align}
 E(X) &= 0.5\times[ 0.1 \times 90 + 0.9 \times (-10)] + 0.5 \times [0.1 \times 30 + 0.01 \times 480 + 0.89 \times (-20)] \\
         &= 0.5\times[9-9] + 0.5\times[3 + 4.8 -17.8] \\
         &= -5
 \end{align}

We can see that $E(X)$ is the mean of the net payout of each lottery because the probability for each lottery is $0.5$. And the net payout of each lottery does not change this fact.


Let $X_1$ denote the payout for the first lottery and $X_2$ for the second.\\\\
Proof:
 \begin{align}
 E(X) &= 0.5 \times E(X_1) + 0.5 E(X_2) \\
         &= \frac{E(X_1) + E(X_2)}{2}
 \end{align}

By definition, this is the mean.

\end{document}